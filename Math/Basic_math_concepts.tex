\documentclass[12pt, svgnames]{article}
\usepackage{amsmath}
\usepackage{mathtools}
\usepackage{pst-node, auto-pst-pdf}
\usepackage{pgf, tikz} 
% \pgfmathsetseed{\number\pdfrandomseed} % to ensure that it is randomized use \randomseed for xelatex

\newcommand{\thecmd}[1]{% 
\pgfmathsetmacro{\a}{int(#1-#1/4)}%
\pgfmathsetmacro{\b}{int(#1+#1/4)}% 
\pgfmathsetmacro{\thenum}{int(random(\a,\b))}%
\thenum%
}%

\begin{document}

\section{Elementray school, Grade 1 - 5}

\subsection*{Two-digit number $\times$ two-digit number}

\begin{itemize}
\item Method 1
\begin{postscript}
    \begin{align*}
        23\times 32 & = (\rnode{l1}20 + \rnode{l2}3)\times\rnode{r1}32 \\
                    & \\
                    & = (20 \times 32) + (3 \times 32) \\
                    & = 640 + 96 \\
                    & = 736
    \end{align*}
    \psset{arrows=->, nodesep=2pt, arcangle=45}
    \ncarc{l1}{r1}
    \psset{arrows=->, nodesep=2pt, arcangle=-45, linecolor=Red}
    \ncarc{l2}{r1}
\end{postscript}

\item Method 2
\begin{postscript}
    \begin{align*}
        23\times 32 & = (\rnode{l1}20 + \rnode{l2}3)\times(\rnode{r1}30 + \rnode{r2}2) \\
                    & \\
                    & = (20 \times 30) + (20 \times 2) + (3 \times 30) + (3 \times 2) \\
                    & = 600 + 40 + 90 + 6 \\
                    & = 736
    \end{align*}
    \psset{arrows=->, nodesep=2pt, arcangle=45}
    \ncarc{l1}{r1}\ncarc[border=.5pt]{l1}{r2}
    \psset{arrows=->, nodesep=2pt, arcangle=-45, linecolor=Red}
    \ncarc{l2}{r1} \ncarc[border=.5pt]{l2}{r2}
\end{postscript}
\end{itemize}


% example
% \Large
% \begin{postscript}
%   \begin{align*}
%     (\rnode{X1}{2x}+\rnode{Y1}{\smash[b]{3y}})(\rnode{X2}{4x}+\rnode{Y2}{5y}) & = (2x)(4x)+(2x)(5y)+(3y)(4x)+(3y)(5y) \\
%                                                                               & = 8x²+10xy+12xy+15y² \\
%                                                                               & = 8x²+22xy+15y²
%   \end{align*}
%   \psset{angle=90,nodesep=2pt, arrows=->, arrowinset=0.2}
%   \ncbar[arm=15pt]{Y1}{Y2}\ncbar[border=1.5pt]{X1}{X2}
%   \psset{angle=-90,nodesep=4pt}
%   \ncbar{X1}{Y2}\ncbar[arm=8pt]{Y1}{X2}
% \end{postscript}

\subsection*{Special cases of two two-digit numbers multiplication}

\begin{itemize}
\item When both two-digit numbers have 1 in the tens place
\item Two same two-digit nunbers that have 5 in the ones place
\end{itemize}


\subsection*{Division, decimal, and fraction}


\subsection*{Scientific notation (for large numbers)}
\begin{align*}
    & 100 = 10^2 \\
    & 1,000 = 10^3 \\
    & 1,000,000 = 10^6 \\
    & 1,200 = 1.2\times 1,000 = 1.2\times 10^3 \\
    & 1,234 = 1.234 \times 10^3 \\
    & 10^4 \times 10^3 = 10^{4+3} = 10^7 \\
    & 10^4 \div 10^3 = {10000\over1000} = 10^{4-3} = 10^1 = 10
\end{align*}

\subsection*{Exercise}

% \thecmd{100} $\times$ \thecmd{10}


\end{document}